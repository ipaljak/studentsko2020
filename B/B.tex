%%%%%%%%%%%%%%%%%%%%%%%%%%%%%%%%%%%%%%%%%%%%%%%%%%%%%%%%%%%%%%%%%%%%%%
% Problem statement
\begin{statement}[
  timelimit=1 s,
  memorylimit=512 MiB,
]{B: Brzi Biljar}

Biljarski stol kvadratnog oblika nalazi se u koordinatnoj ravnini, a njegovi
vrhovi imaju koordinate $(L,L)$, $(L,-L)$, $(-L,L)$ i $(-L,-L)$. Trenutno se
na stolu u točki $(x_1, y_1)$ nalazi i miruje biljarska kugla, dok se u točki
$(x_2, y_2)$ nalazi rupa. Za kuglu i rupu vrijedi da nisu na rubu stola te da
se nalaze na različitim pozicijama.

Udarimo li kuglu, ona će se početi kretati pravocrtno. Ako dođe do ruba stola,
odbija se tako da je kut upada jednak kutu refleksije te se nastavlja kretati
pravocrtno. Staje tek kada se nađe u jednom od četiriju vrhova stola ili u
rupi.

Koristeći svoju veliku snagu, gospodin Malnar je jednom prilikom toliko jako
udario kuglu da nitko osim njega nije uspio vidjeti putanju kugle. Jedino
znano je da je kugla završila u rupi, a preživjeli svjedoci dodatno tvrde da
su pomoću frekvencije zvuka nastalog zbog brzog udaranja kugle mogli
zaključiti da se kugla tijekom svoje putanje od ruba stola odbila najviše $n$
puta.

\begin{figure}[h]
\centering
\includegraphics[width=0.75\textwidth]{img/biljar.png}
\caption{Slika prikazuje sve moguće putanje za prvi probni primjer kada je $k=1$.}
\end{figure}

Analitičare zanima na koje se sve načine kugla mogla kretati. Odredite, za
svaki cijeli broj $0 \le k \le n$, koliko postoji različitih putanja kugle na
kojima se ona od ruba stola odbila točno $k$ puta. Moguće je dokazati da su
svi odgovori konačni brojevi koji stanu u $32$-bitni tip podatka.

%%%%%%%%%%%%%%%%%%%%%%%%%%%%%%%%%%%%%%%%%%%%%%%%%%%%%%%%%%%%%%%%%%%%%%
% Input
\subsection*{Ulazni podaci}
U prvom su retku brojevi $L$ $(2 \le L \le 1\,000\,000)$ i $n$ $(1 \le n \le
500)$ iz teksta zadatka.

U drugom su retku cijeli brojevi $x_1$, $y_1$, $x_2$, $y_2$ $(-L < x_1,
y_1, x_2, y_2 < L)$ iz teksta zadatka. Vrijedi da $(x_1, y_1) \ne (x_2, y_2)$.

%%%%%%%%%%%%%%%%%%%%%%%%%%%%%%%%%%%%%%%%%%%%%%%%%%%%%%%%%%%%%%%%%%%%%%
% Output
\subsection*{Izlazni podaci}
Ispišite $n+1$ brojeva odvojenih razmakom koji redom, od $k=0$ do $k=n$,
predstavljaju broj različitih putanja kugle uz točno $k$ odbijanja.

%%%%%%%%%%%%%%%%%%%%%%%%%%%%%%%%%%%%%%%%%%%%%%%%%%%%%%%%%%%%%%%%%%%%%%
% Examples
\subsection*{Probni primjeri}
\begin{tabularx}{\textwidth}{X'X}
  \textbf{ulaz}
  \linespread{1}{\verbatiminput{test/B.dummy.in.1}}
  \textbf{izlaz}
  \linespread{1}{\verbatiminput{test/B.dummy.out.1}} &
  \textbf{ulaz}
  \linespread{1}{\verbatiminput{test/B.dummy.in.2}}
  \textbf{izlaz}
  \linespread{1}{\verbatiminput{test/B.dummy.out.2}}
\end{tabularx}

%%%%%%%%%%%%%%%%%%%%%%%%%%%%%%%%%%%%%%%%%%%%%%%%%%%%%%%%%%%%%%%%%%%%%%
% We're done
\end{statement}

%%% Local Variables:
%%% mode: latex
%%% mode: flyspell
%%% ispell-local-dictionary: "croatian"
%%% TeX-master: "../studentsko2018.tex"
%%% End:
