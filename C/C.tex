%%%%%%%%%%%%%%%%%%%%%%%%%%%%%%%%%%%%%%%%%%%%%%%%%%%%%%%%%%%%%%%%%%%%%%
% Problem statement
\begin{statement}[
  timelimit=1.5 s,
  memorylimit=512 MiB,
]{C: Carska Civilizacija}

Nakon apokalipse došlo je vrijeme za uzdizanje nove civilizacije predvođene
gospodinom Malnarom. Prvo što će napraviti je obnoviti tramvajsku mrežu u
glavnoj ulici svoga carstva.

Određeno je $n$ potencijalnih mjesta uzduž ulice, a gospodin Malnar izabrat će
neka od njih na kojima će izgraditi tramvajske postaje. Već je određeno da će
se na prvom i $n$-tom mjestu nalaziti okretišta pa ta mjesta svakako moraju
biti odabrana.

Za svako mjesto poznate su vrijednosti $x_i$ i $c_i$, gdje $x_i$ predstavlja
udaljenost tog mjesta od početka ulice, a $c_i$ predstavlja nezadovoljstvo
stanovnika kada bi postaja bila na tom mjestu. Naime, kod nekih postaja
nalaze se kiosci čiji prodavači uvijek podvale skupe karte, a kod nekih
odlične pekare.

Postoji ukupno $m$ stanovnika koji će koristiti planiranu liniju pa je gospodin
Malnar odlučio pitati svakog od njih za mišljenje. Saznao je da $i$-ti
stanovnik mrzi vožnje duljine $d_i$ te ocjenjuje izbor postaja na sljedeći
način:

\begin{itemize}
    \item ukupno zadovoljstvo tog stanovnika jednako je zbroju zadovoljstva za
          svaki par uzastopnih izabranih postaja
    \item zadovoljstvo sa dvije uzastopne izabrane postaje koje su udaljene za
          $d$ jednako je $|d - d_i|$
\end{itemize}

Ukupno zadovoljstvo gospodin Malnar računa tako da zbroji zadovoljstva svih $m$
stanovnika te na kraju oduzme nezadovoljstva $c_k$ svih postaja $k$ koje su
odabrane. Ispišite maksimalno zadovoljstvo koje gospodin Malnar može postići
uz optimalan izbor postaja.

%%%%%%%%%%%%%%%%%%%%%%%%%%%%%%%%%%%%%%%%%%%%%%%%%%%%%%%%%%%%%%%%%%%%%%
% Input
\subsection*{Ulazni podaci}
 U prvom su retku prirodni brojevi $n$ $(2 \le n \le 100\,000)$ i $m$
$(1 \le m \le 100\,000)$ iz teksta zadatka.

Slijedi redak s $m$ brojeva $d_i$ $(0 \le d_i \le 10^7)$ iz teksta zadatka.

Slijedi $n$ redaka koji opisuju potencijalne postaje. Svaka postaja opisana je
pozicijom $x_i$ $(0 \le x_i \le 10^7)$ i nezadovoljstvom $c_i$ $(0 \le |c_i|
\le 10^{12})$. Dodatno vrijedi $x_1 < x_2 < \ldots < x_n$.

%%%%%%%%%%%%%%%%%%%%%%%%%%%%%%%%%%%%%%%%%%%%%%%%%%%%%%%%%%%%%%%%%%%%%%
% Output
\subsection*{Izlazni podaci}
U jedinom retku ispišite najveće moguće zadovoljstvo.

%%%%%%%%%%%%%%%%%%%%%%%%%%%%%%%%%%%%%%%%%%%%%%%%%%%%%%%%%%%%%%%%%%%%%%
% Examples
\subsection*{Probni primjeri}
\begin{tabularx}{\textwidth}{X'X'X}
  \textbf{ulaz}
  \linespread{1}{\verbatiminput{test/C.dummy.in.1}}
  \textbf{izlaz}
  \linespread{1}{\verbatiminput{test/C.dummy.out.1}} &
  \textbf{ulaz}
  \linespread{1}{\verbatiminput{test/C.dummy.in.2}}
  \textbf{izlaz}
  \linespread{1}{\verbatiminput{test/C.dummy.out.2}} &
  \textbf{ulaz}
  \linespread{1}{\verbatiminput{test/C.dummy.in.3}}
  \textbf{izlaz}
  \linespread{1}{\verbatiminput{test/C.dummy.out.3}}
\end{tabularx}

\textbf{Pojašnjenje trećeg probnog primjera:}
Optimalno je izabrati postaje $1$, $3$, $5$, $7$ i $9$.  Uzastopne udaljenosti
tada su $6$, $17$, $49$ i $25$, a zadovoljstva stanovnika redom $61$, $159$,
$89$, $275$ i $171$.  Zbroj njihovih zadovoljstava je $755$, a ukupno
nezadovoljstvo izbora je $618$ pa je konačni rezultat $137$.

%%%%%%%%%%%%%%%%%%%%%%%%%%%%%%%%%%%%%%%%%%%%%%%%%%%%%%%%%%%%%%%%%%%%%%
% We're done
\end{statement}

%%% Local Variables:
%%% mode: latex
%%% mode: flyspell
%%% ispell-local-dictionary: "croatian"
%%% TeX-master: "../studentsko2018.tex"
%%% End:
