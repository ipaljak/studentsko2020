%%%%%%%%%%%%%%%%%%%%%%%%%%%%%%%%%%%%%%%%%%%%%%%%%%%%%%%%%%%%%%%%%%%%%%
% Problem statement
\begin{statement}[
  timelimit=1 s,
  memorylimit=512 MiB,
]{H: Hvalevrijedan Hitac}

Ovogodišnju Noć vještica nećemo pamtiti po impresivno izrezbarenim bundevama
ili razuzdanim tulumima. Nažalost, pamtit ćemo je po odlasku Seana Connerya,
poznatog škotskog glumca koji je prvi utjelovio lik Jamesa Bonda.  Dakako,
ova je vijest odjeknula svijetom te su se mnoge slavne osobe od ovog velikana
oprostile putem društvenih mreža. Tako je Pierce Brosnan izjavio da je za
njega Sean Connery najbolji James Bond, a Josip Manolić se pomalo našalio
rekavši da mu je najteže kada kolege špijuni odlaze mladi.  Zanimljivo,
gospodin Malnar nije se oglasio na svom facebook profilu, već je je istog
trena posegnuo za videokasetom svog najdražeg filma. Dakako, radi se o filmu
\textit{Nedodirljivi} iz 1987.\ u kojem lik Seana Connerya izgovara poznatu
rečenicu -- ``\textit{Never bring a knife to a gunfight.}''.

Gospodin Malnar je zbog svojih vještina bacanja noževa oduvijek bio sumnjičav
prema toj tvrdnji te je odlučio provjeriti njenu istinitost postavivši
mete na stablo koje mu se nalazi u dvorištu. Brzo je odredio da se stablo
sastoji od $n$ čvorova te je u svaki čvor stabla odlučio postaviti jednu
metu. Mete je izradio sam, bile su kružnog oblika te je jedna strana mete bila
obojena u zelenu, a druga u crvenu boju. Nakon što je postavio mete u čvorove
stabla, krenuo ih je gađati noževima.

Ubrzo je primijetio da je nepogrešiv, odnosno, savršenom će preciznošću
pogoditi upravo onu metu koju je gađao. Dodatno, zbog siline hitca ta će meta
biti \textbf{potpuno uništena}, a mete koje se nalaze na susjednim čvorovima
toj meti će se zaokrenuti te, iz perspektive gospodina Malnara, promijeniti
boju. Oduševljen ovom činjenicom, gospodin Malnar osmislio je sljedeću igru:

\begin{itemize}
    \item Na početku igre u svakom se čvoru stabla nalazi meta.
    \item Gospodin Malnar jednim hitcem smije pogoditi neku od meta koje su,
          iz njegove perspektive, obojene zelenom bojom. Sukladno prethodnom
          odlomku, pogođena meta će biti uništena, a njoj susjedne mete će
          promijeniti boju.
    \item Cilj igre je uništiti sve mete koje se nalaze na stablu.
\end{itemize}

Vaš je zadatak odrediti može li gospodin Malnar pobijediti u ovoj igri te, ako
može, kojim redom treba gađati mete kako bi mu to pošlo za rukom.

\textbf{Napomena:} Dvije su mete susjedne ako se nalaze u čvorovima koji su
direktno spojeni bridom.

%%%%%%%%%%%%%%%%%%%%%%%%%%%%%%%%%%%%%%%%%%%%%%%%%%%%%%%%%%%%%%%%%%%%%%
% Input
\subsection*{Ulazni podaci}
U prvom je retku prirodan broj $n$ $(1 \le n \le 200\,000)$ iz teksta zadatka.

U drugom je retku $n$ brojeva odvojenih razmacima koji predstavljaju boje meta
koje se nalaze u čvorovima stabla. Preciznije, $i$-ti broj u drugom retku je $1$
ako gospodin Malnar u $i$-tom čvoru vidi metu zelene boje, odnosno $0$ ako u tom
čvoru vidi metu crvene boje.

U sljedećih se $n-1$ redaka nalaze po dva prirodna broja $a$ i $b$ $(1 \le a, b
\le n)$ koji predstavljaju brid između čvorova $a$ i $b$. Bridovi su takvi da
čine stablo, jednostavan povezan graf bez ciklusa.

%%%%%%%%%%%%%%%%%%%%%%%%%%%%%%%%%%%%%%%%%%%%%%%%%%%%%%%%%%%%%%%%%%%%%%
% Output
\subsection*{Izlazni podaci}
U prvi redak ispišite riječ \texttt{POBJEDA} ako gospodin Malnar može
pobijediti u igri, odnosno \texttt{PORAZ} ako to ne može.

U slučaju da je pobjeda moguća, u drugom je retku potrebno ispisati $n$
prirodnih brojeva koji redom označavaju hitce gospodina Malnara. Odnosno,
$i$-ti ispisani broj treba predstavljati oznaku čvora u kojem se nalazi meta
koju će gospodin Malnar uništiti $i$-tim hitcem.

%%%%%%%%%%%%%%%%%%%%%%%%%%%%%%%%%%%%%%%%%%%%%%%%%%%%%%%%%%%%%%%%%%%%%%
% Examples
\subsection*{Probni primjeri}
\begin{tabularx}{\textwidth}{X'X}
  \textbf{ulaz}
  \linespread{1}{\verbatiminput{test/H.dummy.in.1}}
  \textbf{izlaz}
  \linespread{1}{\verbatiminput{test/H.dummy.out.1}} &
  \textbf{ulaz}
  \linespread{1}{\verbatiminput{test/H.dummy.in.2}}
  \textbf{izlaz}
  \linespread{1}{\verbatiminput{test/H.dummy.out.2}}
\end{tabularx}

%%%%%%%%%%%%%%%%%%%%%%%%%%%%%%%%%%%%%%%%%%%%%%%%%%%%%%%%%%%%%%%%%%%%%%
% We're done
\end{statement}

%%% Local Variables:
%%% mode: latex
%%% mode: flyspell
%%% ispell-local-dictionary: "croatian"
%%% TeX-master: "../studentsko2018.tex"
%%% End:
