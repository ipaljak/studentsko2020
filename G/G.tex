%%%%%%%%%%%%%%%%%%%%%%%%%%%%%%%%%%%%%%%%%%%%%%%%%%%%%%%%%%%%%%%%%%%%%%
% Problem statement
\begin{statement}[
  timelimit=1 s,
  memorylimit=512 MiB,
]{G: Gospodar Gljiva}

Gospodin Malnar odlučio je ove godine organizirati novogodišnju proslavu na
koju će pozvati svojih $n$ najboljih prijatelja. Budući da se radi o
\textbf{najluđoj noći} u godini, svakom će prijatelju pokloniti jednu gljivu
pomoću koje će taj prijatelj naručenu pizzu margheritu pretvoriti u capricciosu.

Gospodin Malnar inače posjeduje beskonačno mnogo gljiva, a svaku je od njih
označio različitim nenegativnim cijelim brojem. Prije početka same zabave,
gljive će staviti u vreću iz koje će svaki gost izvući svoju gljivu.
Nažalost, nije uspio nabaviti dovoljno veliku vreću u koju bi stale sve
gljive i sada nikako ne može odrediti koje će gljive staviti u vreću. Nakon
što je još malo razmislio, donio je sljedeću odluku:

\vspace{-2.5mm}
\begin{itemize}
    \item Prije početka zabave u vreći će se nalaziti točno $n$ gljiva.
    \item Ako se u vreći nalazi gljiva s oznakom $x > 0$, tada se u vreći mora
      nalaziti i gljiva s oznakom $\lfloor \frac{x-1}{k} \rfloor$.
\end{itemize}
\vspace{-2.5mm}

Pomozite gospodinu Malnaru i odredite na koliko različitih načina može
pripremiti vreću gljiva za novogodišnju zabavu.

\textbf{Napomena:} Budući da traženi broj načina može biti vrlo velik, potrebno
je samo ispisati njegov ostatak pri djeljenju s $10^9+7$.

%%%%%%%%%%%%%%%%%%%%%%%%%%%%%%%%%%%%%%%%%%%%%%%%%%%%%%%%%%%%%%%%%%%%%%
% Input
\subsection*{Ulazni podaci}
U prvom su retku prirodni brojevi $n$ $(2 \le n \le 1\,000\,000)$ i $k$
$(1 \le k \le 1\,000\,000)$.

%%%%%%%%%%%%%%%%%%%%%%%%%%%%%%%%%%%%%%%%%%%%%%%%%%%%%%%%%%%%%%%%%%%%%%
% Output
\subsection*{Izlazni podaci}
U prvom retku ispišite traženi broj načina modulo $10^9 + 7$.

%%%%%%%%%%%%%%%%%%%%%%%%%%%%%%%%%%%%%%%%%%%%%%%%%%%%%%%%%%%%%%%%%%%%%%
% Examples
\subsection*{Probni primjeri}
\begin{tabularx}{\textwidth}{X'X}
  \textbf{ulaz}
  \linespread{1}{\verbatiminput{test/G.dummy.in.1}}
  \textbf{izlaz}
  \linespread{1}{\verbatiminput{test/G.dummy.out.1}} &
  \textbf{ulaz}
  \linespread{1}{\verbatiminput{test/G.dummy.in.2}}
  \textbf{izlaz}
  \linespread{1}{\verbatiminput{test/G.dummy.out.2}}
\end{tabularx}

\textbf{Pojašnjenje prvog probnog primjera:} moguće vreće su $\{0, 1, 2\}$, $\{0, 1, 3\}$, $\{0, 1, 4\}$, $\{0, 2, 5\}$ i $\{0, 2, 6\}$

%%%%%%%%%%%%%%%%%%%%%%%%%%%%%%%%%%%%%%%%%%%%%%%%%%%%%%%%%%%%%%%%%%%%%%
% We're done
\end{statement}

%%% Local Variables:
%%% mode: latex
%%% mode: flyspell
%%% ispell-local-dictionary: "croatian"
%%% TeX-master: "../studentsko2018.tex"
%%% End:
