%%%%%%%%%%%%%%%%%%%%%%%%%%%%%%%%%%%%%%%%%%%%%%%%%%%%%%%%%%%%%%%%%%%%%%
% Problem statement
\begin{statement}[
  timelimit=1 s,
  memorylimit=512 MiB,
]{D: Drugi Dio}

Naš omiljeni vodič shvatio je da je u \textbf{prvom dijelu} izvedena spačka
koja je uništila njegove šanse za pobjedu. Poražen, posramljen i prašnjav,
vodič je skovao plan za osvetu. Ovaj put je prednost domaćeg terena na njegovoj
strani, a gospodin Malnar naizgled nema šanse. Naime, radi se o utrci!
Nije to utrka na 100 metara ili neki pišljivi maraton, ovo je epska utrka
između dvaju gradova u zemlji vatre. Jedino pravlio je da nema pravila, a sve
što je važno jest stići iz polazišnog grada u odredišni grad prije suparnika.

Vodič se odlučio utrkivati biciklom jer zna da su trenutno zatvorene sve
međugradske prometnice za automobile. Budući da zna da gospodin Malnar nije
svjestan te činjenice te se smatra fizički nadmoćnijim, dopustit će mu da sam
odabere između kojih će se gradova utrkivati.

Međutim, vodič ne zna da je gospodin Malnar ionako već unajmio privatni
helikopter u slučaju da treba obaviti neke stvari na drugom kraju grada.
Naravno, gospodin Malnar je prihvatio izazov, ali se i malo sažalio nad
jadnim vodičem pa je odlučio odabrati rutu u kojoj će pobijediti s najmanjom
mogućom prednosti.

Gradovi u Azerbajdžanu mogu se prikazati kao točke u koordinatnom sustavu.
Međugradske biciklističke staze poprimaju oblik pravokutne mreže pa se vodič
može kretati samo usporedno s koordinatnim osima. S druge pak strane,
gospodin Malnar će se između gradova kretati dužinom koja ih spaja.
Preciznije, ako se početni grad nalazi u točki $(x_1, y_1)$, a završni se
nalazi u točki $(x_2, y_2)$, vodič će prevaliti udaljenost $dv = |x_1 - x_2|
+ |y_1 - y_2|$, dok će gospodin Malnar prevaliti udaljenost $dm =
\sqrt{(x_1-x_2)^2 + (y_1-y_2)^2}$.

Gospodin Malnar će odabrati par gradova za koje je omjer $\frac{dv}{dm}$
najmanji mogući. Odredite taj omjer!

%%%%%%%%%%%%%%%%%%%%%%%%%%%%%%%%%%%%%%%%%%%%%%%%%%%%%%%%%%%%%%%%%%%%%%
% Input
\subsection*{Ulazni podaci}
U prvom je retku prirodan broj $n$ $(2 \le n \le 300\,000)$ iz teksta
zadatka.

U $i$-tom od sljedećih $n$ redaka nalaze se cijeli brojevi $x_i$, $y_i$
$(0 \le |x_i|, |y_i| \le 10^9)$ koji predstavljaju koordinate $i$-tog
grada. Niti jedna dva grada neće se nalaziti na istim koordinatama.

%%%%%%%%%%%%%%%%%%%%%%%%%%%%%%%%%%%%%%%%%%%%%%%%%%%%%%%%%%%%%%%%%%%%%%
% Output
\subsection*{Izlazni podaci}
U prvom je retku potrebno ispisati traženi omjer iz teksta zadatka.

Tolerirat će se apsolutno ili relativno odstupanje od službenog rješenja
za $10^{-10}$.

%%%%%%%%%%%%%%%%%%%%%%%%%%%%%%%%%%%%%%%%%%%%%%%%%%%%%%%%%%%%%%%%%%%%%%
% Examples
\subsection*{Probni primjeri}
\begin{tabularx}{\textwidth}{X'X}
  \textbf{ulaz}
  \linespread{1}{\verbatiminput{test/D.dummy.in.1}}
  \textbf{izlaz}
  \linespread{1}{\verbatiminput{test/D.dummy.out.1}} &
  \textbf{ulaz}
  \linespread{1}{\verbatiminput{test/D.dummy.in.2}}
  \textbf{izlaz}
  \linespread{1}{\verbatiminput{test/D.dummy.out.2}}
\end{tabularx}

%%%%%%%%%%%%%%%%%%%%%%%%%%%%%%%%%%%%%%%%%%%%%%%%%%%%%%%%%%%%%%%%%%%%%%
% We're done
\end{statement}

%%% Local Variables:
%%% mode: latex
%%% mode: flyspell
%%% ispell-local-dictionary: "croatian"
%%% TeX-master: "../studentsko2018.tex"
%%% End:
