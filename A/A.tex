%%%%%%%%%%%%%%%%%%%%%%%%%%%%%%%%%%%%%%%%%%%%%%%%%%%%%%%%%%%%%%%%%%%%%%
% Problem statement
\begin{statement}[
  timelimit=1 s,
  memorylimit=512 MiB,
]{A: ASCII Art}

Gospodin Malnar strastveni je zaljubljenik u umjetnost i urbanu kulturu grada
Zagreba, stoga ne čudi što je iz godine u godinu stalan gost manifestacije
\textit{Art Park} koja se ove godine održala u parku Ribnjak. Zanimljivo da
je upravo tamo dobio inspiraciju za ovaj zadatak. Naime, razledavajući
remek-djela izložbe ``\textit{Kauboji, pištolji i feminizam}'', upoznao je
jednu mladu djevojku.

\textbf{Gospodin Malnar:} Primjećuješ li kako suvremeni umjetnici vrlo
rijetko posežu za \textit{ASCII art} tehnikom.

\textbf{Djevojka:} Moram priznati da nisam upoznata s tom tehnikom. O čemu se
točno radi?

\textbf{Gospodin Malnar:} To je tehnika pomoću koje umjetnici prikazuju vrlo
kompleksne slike koristeći 128 znakova definiranih ASCII standardom. Ako
želiš, pokazat ću ti neke svoje uratke, a usput bih te mogao počastiti i
sokom od hmelja.

\textbf{Djevojka:} Zvući zanimljivo, može!

U ravnini je istaknuto $n$ cjelobrojnih točaka, a vaš je zadatak nacrtati ih u
koordinatnom sustavu koristeći \textit{ASCII art} tehniku.

Svaku od istaknutih točaka na slici je potrebno predstaviti znakom
\texttt{'x'} (ASCII $120$). Ako se među istaknutim točkama ne nalazi
ishodište koordinatnog sustava, tada ga je potrebno predstaviti znakom
\texttt{'o'} (ASCII $111$).  Također je posebnim znakovima potrebno
predstaviti dijelove koordinatnih osi na kojima se ne nalaze istaknute točke.
Preciznije, znakom \texttt{'-'} (ASCII $45$) potrebno je predstaviti takve
dijelove $x$-osi, a znakom \texttt{'|'} (ASCII $124$) potrebno je predstaviti
takve dijelove $y$-osi.  Preostale dijelove ravnine na kojima se ne nalazi
niti jedna istaknuta točka, ishodište ili koordinatna os, potrebno je
predstaviti znakom praznine \texttt{' '} (ASCII $32$).

Dodatno, cijelu je sliku potrebno smjestiti u pravokutni okvir \textbf{najmanje
moguće površine} čiji rub na slici treba biti označen znakovima \texttt{'\#'}
(ASCII $35$). Dakako, unutar okvira moraju se nalaziti sve istaknute točke
zajedno s ishodištem.

Primijetite da navedeni zahtjevi jednoznačno određuju izgled slike.

%%%%%%%%%%%%%%%%%%%%%%%%%%%%%%%%%%%%%%%%%%%%%%%%%%%%%%%%%%%%%%%%%%%%%%
% Input
\subsection*{Ulazni podaci}
  U prvom se retku nalazi prirodan broj $n$ $(1 \le n \le 5\,000)$ iz teksta
zadatka.

U $i$-tom od sljedećih $n$ redaka nalaze se po dva cijela broja $x_i$ i
$y_i$ $(-500 \le x_i, y_i \le 500)$ koji predstavljaju koordinate $i$-te
istaknute točke. Svaka će se točka u ulazu pojaviti najviše jednom.

%%%%%%%%%%%%%%%%%%%%%%%%%%%%%%%%%%%%%%%%%%%%%%%%%%%%%%%%%%%%%%%%%%%%%%
% Output
\subsection*{Izlazni podaci}
Potrebno je ispisati \textit{ASCII art} sliku koordinatnog sustava s
istaknutim točkama kako je opisano u tekstu zadatka.

%%%%%%%%%%%%%%%%%%%%%%%%%%%%%%%%%%%%%%%%%%%%%%%%%%%%%%%%%%%%%%%%%%%%%%
% Examples
\subsection*{Probni primjeri}
\begin{tabularx}{\textwidth}{X'X'X}
  \textbf{ulaz}
  \linespread{1}{\verbatiminput{test/A.dummy.in.1}}
  \textbf{izlaz}
  \linespread{1}{\verbatiminput{test/A.dummy.out.1}} &
  \textbf{ulaz}
  \linespread{1}{\verbatiminput{test/A.dummy.in.2}}
  \textbf{izlaz}
  \linespread{1}{\verbatiminput{test/A.dummy.out.2}} &
  \textbf{ulaz}
  \linespread{1}{\verbatiminput{test/A.dummy.in.3}}
  \textbf{izlaz}
  \linespread{1}{\verbatiminput{test/A.dummy.out.3}}
\end{tabularx}

%%%%%%%%%%%%%%%%%%%%%%%%%%%%%%%%%%%%%%%%%%%%%%%%%%%%%%%%%%%%%%%%%%%%%%
% We're done
\end{statement}

%%% Local Variables:
%%% mode: latex
%%% mode: flyspell
%%% ispell-local-dictionary: "croatian"
%%% TeX-master: "../studentsko2018.tex"
%%% End:
