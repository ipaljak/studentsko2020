%%%%%%%%%%%%%%%%%%%%%%%%%%%%%%%%%%%%%%%%%%%%%%%%%%%%%%%%%%%%%%%%%%%%%%
% Problem statement
\begin{statement}[
  timelimit=1 s,
  memorylimit=512 MiB,
]{E: Ekstremna Ekspedicija}

Mladi informatičar Kile mora skidati kile te se stoga odlučio uputiti na
Sljeme.  Proučavajući planinarske karte, Kile je primijetio da staze na
Sljemenu čine stablastu strukturu. Preciznije, poistovjetio ih je s bridovima u
stablu, dok je mjesta na kojima se staze sijeku predstavio čvorovima.

Zaključio je da se stablo sastoji od $n$ čvorova koje je označio prirodnim
brojevima od $1$ od $n$. Zatim je isplanirao $q$ izleta, gdje $i$-ti izlet
počinje u čvoru $a_i$, a završava u čvoru $b_i$. Također je pomalo nadobudno
procijenio da će udaljenost između bilo koja dva susjedna čvora prevaliti
za točno jednu minutu.

Međutim, Kile nije naročito poznat po svojim orijentacijskim vještinama. Stoga
će, nakon što se nađe u nekom čvoru, nasumično i uniformno odabrati sljedeću
stazu kojom će kročiti (među stazama kojima je taj čvor jedna od krajnjih
točaka). Kako bi Kile mogao isplanirati svoje daljnje aktivnosti, za svaki od
$q$ izleta zanima ga očekivano vrijeme koje će provesti pentrajući se po
Sljemenu. Odnosno, zanima ga koliko je očekivano vrijeme (u minutama)
prolaska od čvora $a_i$ do čvora $b_i$ ako će se kretati na gore opisan
način. Pomozite mu!

\textbf{Napomena:} Moguće je dokazati da se traženo očekivano vrijeme može
zapisati u obliku neskrativog razlomka $\frac{P}{R}$. Da bismo izbjegli
probleme s preciznošću, potrebno je ispisati broj $P \cdot R^{-1}
\Mod{10^9+7}$

%%%%%%%%%%%%%%%%%%%%%%%%%%%%%%%%%%%%%%%%%%%%%%%%%%%%%%%%%%%%%%%%%%%%%%
% Input
\subsection*{Ulazni podaci}
U prvom su retku prirodni brojevi $n$ i $q$ $(2 \le n, q \le 300\,000)$
iz teksta zadatka.

U sljedećih $n-1$ redaka nalaze se brojevi $u_i$ i $v_i$ $(1 \le u_i, v_i \le
n)$ koji označavaju da su čvorovi s oznakama $u_i$ i $v_i$ direktno povezani
bridom.  Bridovi će biti takvi da tvore stablo (jednostavan povezan graf bez
ciklusa) od $n$ čvorova.

U $i$-tom od sljedećih $q$ redaka nalaze se međusobno različiti brojevi $a_i$ i
$b_i$ $(1 \le a_i, b_i \le n)$ koji predstavljaju početnu i završnu
točku $i$-tog izleta.

%%%%%%%%%%%%%%%%%%%%%%%%%%%%%%%%%%%%%%%%%%%%%%%%%%%%%%%%%%%%%%%%%%%%%%
% Output
\subsection*{Izlazni podaci}
U $i$-tom je retku potrebno ispisati očekivano trajanje $i$-tog izleta kako
je opisano u tekstu zadatka.

%%%%%%%%%%%%%%%%%%%%%%%%%%%%%%%%%%%%%%%%%%%%%%%%%%%%%%%%%%%%%%%%%%%%%%
% Examples
\subsection*{Probni primjeri}
\begin{tabularx}{\textwidth}{X'X'X}
  \textbf{ulaz}
  \linespread{1}{\verbatiminput{test/E.dummy.in.1}}
  \textbf{izlaz}
  \linespread{1}{\verbatiminput{test/E.dummy.out.1}} &
  \textbf{ulaz}
  \linespread{1}{\verbatiminput{test/E.dummy.in.2}}
  \textbf{izlaz}
  \linespread{1}{\verbatiminput{test/E.dummy.out.2}} &
  \textbf{ulaz}
  \linespread{1}{\verbatiminput{test/E.dummy.in.3}}
  \textbf{izlaz}
  \linespread{1}{\verbatiminput{test/E.dummy.out.3}}
\end{tabularx}

%%%%%%%%%%%%%%%%%%%%%%%%%%%%%%%%%%%%%%%%%%%%%%%%%%%%%%%%%%%%%%%%%%%%%%
% We're done
\end{statement}

%%% Local Variables:
%%% mode: latex
%%% mode: flyspell
%%% ispell-local-dictionary: "croatian"
%%% TeX-master: "../studentsko2018.tex"
%%% End:
