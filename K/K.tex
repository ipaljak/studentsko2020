%%%%%%%%%%%%%%%%%%%%%%%%%%%%%%%%%%%%%%%%%%%%%%%%%%%%%%%%%%%%%%%%%%%%%%
% Problem statement
\begin{statement}[
  timelimit=1 s,
  memorylimit=512 MiB,
]{K: Klasična Karantena}

Uslijed globalne pandemije \textit{COVID-19}, nacionalni je stožer civilne
zaštite donio novi niz smjernica i uputa s ciljem prevencije daljnjeg širenja
zaraze među populacijom. Jedna od smjernica odnosi se na obavezno nošenje
zaštitnih maski u svim ugostiteljskim objektima, što uključuje i gostionice,
odnosno birtije.

Na vratima jedne lokalne birtije odmah je osvanuo natpis \textbf{OBAVEZNO
NOŠENJE MASKI!!!}. Međutim, budući da se radi samo o smjernicama, vlasnici
birtije ne mogu natjerati svoje posjetitelje da nose maske. Primijetili su da
se u birtiji trenutno nalazi $a$ ljudi koji nose maske i $b$ ljudi koji ne
nose maske, te im je također poznato da će tijekom večeri u birtiju doći još
$n$ ljudi. Duboko razumijevanje ljudske prirode uz dobro poznavanje vlastitih
mušterija omogučilo je vlasnicima da s nevjerojatnom prezinošću zaključe kako
će $i$-ti novopridošli gost staviti masku ako i samo ako se prije njegovog
ulaska u birtiji nalazi $p_i\%$ ljudi koji nose mase.

Nažalost, vlasnici birtije ne znaju kojim će redoslijedom gosti dolaziti u
birtiju, ali znaju da nitko neće otići. Stoga ih zanima koji je najmanji,
a koji najveći broj ljudi koji će u birtiji nositi maske nakon što uđe svih
$n$ gostiju.

%%%%%%%%%%%%%%%%%%%%%%%%%%%%%%%%%%%%%%%%%%%%%%%%%%%%%%%%%%%%%%%%%%%%%%
% Input
\subsection*{Ulazni podaci}
U prvom se retku nalaze dva cijela broja $a$ i $b$ $(0 \le a, b \le 10^9)$ iz
teksta zadatka.

U drugom se retku nalazi prirodan broj $n$ $(1 \le n \le 500\,000)$ iz teksta
zadatka.

U $i$-tom od sljedećih $n$ redaka nalazi se realan broj $p_i$ $(0 \le p_i \le
100)$ iz teksta zadatka. Svaki od brojeva $p_i$ bit će zapisan na dvije
decimale te će slijediti znak \texttt{'\%'} (ASCII $37$).

%%%%%%%%%%%%%%%%%%%%%%%%%%%%%%%%%%%%%%%%%%%%%%%%%%%%%%%%%%%%%%%%%%%%%%
% Output
\subsection*{Izlazni podaci}
U jednom je retku potrebno ispisati dva prirodna broja koji redom označavaju
najmanji i najveći broj ljudi koji će u birtiji nositi maske nakon što uđe
svih $n$ gostiju.

%%%%%%%%%%%%%%%%%%%%%%%%%%%%%%%%%%%%%%%%%%%%%%%%%%%%%%%%%%%%%%%%%%%%%%
% Examples
\subsection*{Probni primjeri}
\begin{tabularx}{\textwidth}{X'X'X}
  \textbf{ulaz}
  \linespread{1}{\verbatiminput{test/K.dummy.in.1}}
  \textbf{izlaz}
  \linespread{1}{\verbatiminput{test/K.dummy.out.1}} &
  \textbf{ulaz}
  \linespread{1}{\verbatiminput{test/K.dummy.in.2}}
  \textbf{izlaz}
  \linespread{1}{\verbatiminput{test/K.dummy.out.2}} &
  \textbf{ulaz}
  \linespread{1}{\verbatiminput{test/K.dummy.in.3}}
  \textbf{izlaz}
  \linespread{1}{\verbatiminput{test/K.dummy.out.3}}
\end{tabularx}

%%%%%%%%%%%%%%%%%%%%%%%%%%%%%%%%%%%%%%%%%%%%%%%%%%%%%%%%%%%%%%%%%%%%%%
% We're done
\end{statement}

%%% Local Variables:
%%% mode: latex
%%% mode: flyspell
%%% ispell-local-dictionary: "croatian"
%%% TeX-master: "../studentsko2018.tex"
%%% End:
