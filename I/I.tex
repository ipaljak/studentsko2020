%%%%%%%%%%%%%%%%%%%%%%%%%%%%%%%%%%%%%%%%%%%%%%%%%%%%%%%%%%%%%%%%%%%%%%
% Problem statement
\begin{statement}[
  timelimit=1 s,
  memorylimit=512 MiB,
]{I: Izvanredna Isplata}

Međunarodne olimpijade nisu prilika samo natjecateljima da pokažu svoje znanje,
već i gospodinu Malnaru koji željno iščekuje isprobati specijalitete u novoj
državi. Kako bi bio spreman na plaćanje skupocjenih večera, odlučio je prije
puta pretvoriti dio novca u valutu nadolazeće države.

U toj su državi svi iznosi prirodni brojevi te postoji $n$ različitih
vrijednosti kovanica $c_1 < c_2 < \dots < c_n$ koje se koriste za
isplaćivanje iznosa. Novčanik gospodina Malnara možemo zamisliti kao
beskonačan izvor novca, gdje on na raspolaganju ima proizvoljno mnogo
kovanica svake vrijednosti. Kako bi isplatio iznos, gospodin Malnar izabrat
će neki broj kovanica koje u zbroju daju \textbf{točan iznos}. Dodatno
vrijedi $c_1 = 1$, što osigurava da je svaki iznos moguće isplatiti.

Gospodin Malnar se ne zamara previše s izborom kovanica pa koristi sljedeći
pohlepni algoritam za isplaćivanje nekog iznosa -- bira najveću kovanicu koja
ne prelazi iznos koji je potrebno isplatiti, a za preostali dio iznosa
ponavlja ovaj postupak sve dok ga ne isplati do kraja. Budući da gospodin
Malnar ne voli osjećaj prljavog novca u rukama, njemu bi bilo idealno kada bi
svaki mogući iznos njegov pohlepni algoritam isplatio koristeći minimalan
broj kovanica. Takav sustav kovanica gospodin Malnar smatra
\textit{izvanrednim}.

Gospodin Malnar je zasad bio u $t$ država i za svaku od njih poznaje tamošnji
sustav kovanica. Ispišite za svaku državu \texttt{"DA"} ili \texttt{"NE"}
ovisno o tome je li sustav kovanica u toj državi izvanredan.

%%%%%%%%%%%%%%%%%%%%%%%%%%%%%%%%%%%%%%%%%%%%%%%%%%%%%%%%%%%%%%%%%%%%%%
% Input
\subsection*{Ulazni podaci}
U prvom je retku prirodan broj $t$ $(1 \le t \le 100)$ iz teksta zadatka.

Slijedi $t$ opisa država pri čemu je svaka država opisana s dva retka. U prvom
je prirodan broj $n$ $(1 \le n \le 10\,000)$, a u drugom su
prirodni brojevi $1 = c_1 < c_2 < \dots < c_n \le 10\,000$ iz teksta zadatka.
Zbroj svih vrijednosti $n$ po svim državama \textbf{ne prelazi} $10\,000$.

%%%%%%%%%%%%%%%%%%%%%%%%%%%%%%%%%%%%%%%%%%%%%%%%%%%%%%%%%%%%%%%%%%%%%%
% Output
\subsection*{Izlazni podaci}
Ispišite $t$ redaka, za svaku državu odgovor na pitanje je li sustav
kovanica izvanredan.

%%%%%%%%%%%%%%%%%%%%%%%%%%%%%%%%%%%%%%%%%%%%%%%%%%%%%%%%%%%%%%%%%%%%%%
% Examples
\subsection*{Probni primjer}
\begin{tabularx}{\textwidth}{X}
  \textbf{ulaz}
  \linespread{1}{\verbatiminput{test/I.dummy.in.1}}
  \textbf{izlaz}
  \linespread{1}{\verbatiminput{test/I.dummy.out.1}}
\end{tabularx}
\textbf{Pojašnjenje probnog primjera:} u trećoj državi iznos $6$ moguće je
isplatiti koristeći dvije kovanice $(6 = 3 + 3)$, no pohlepni algoritam koristi
tri kovanice $(6 = 4 + 1 + 1)$.

%%%%%%%%%%%%%%%%%%%%%%%%%%%%%%%%%%%%%%%%%%%%%%%%%%%%%%%%%%%%%%%%%%%%%%
% We're done
\end{statement}

%%% Local Variables:
%%% mode: latex
%%% mode: flyspell
%%% ispell-local-dictionary: "croatian"
%%% TeX-master: "../studentsko2018.tex"
%%% End:
